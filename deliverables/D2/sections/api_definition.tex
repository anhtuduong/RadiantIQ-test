\section{API Definition} \label{api_definition}

Creating a well-defined API is crucial for ensuring clear communication between different services. This section will be a guideline for establishing a structured and consistent approach to defining APIs in RadiantIQ project.

The API definition will be documented in GraphQL Schema Definition Language (SDL) format, located in the \texttt{api} directory of the project repository and also published on \href{TODO add link here}{\textcolor{blue}{RadiantIQ GraphQL API documentation}}.

\subsection{General Principles}

\begin{itemize}
    \item \textbf{Consistency:} Ensure that naming conventions, structures, and responses are consistent across all endpoints.
    \item \textbf{Simplicity:} Design the API to be simple and intuitive, making it easy for everyone to understand and use.
    \item \textbf{GraphQL Principles:} Follow GraphQL design principles, including declarative data fetching and single endpoint structure.
    \item \textbf{Documentation:} Document the API comprehensively, including all queries, mutations, parameters, request and response formats, and error codes.
    \item \textbf{Versioning:} Include versioning in the API to allow for backward compatibility and future updates.
    \item \textbf{Security:} Implement appropriate security measures, such as authentication and authorization, to protect the API from unauthorized access.
    \item \textbf{Error Handling:} Define clear error messages and codes to help developers troubleshoot issues.
    \item \textbf{Testing:} Test the API thoroughly to ensure that it functions as expected and returns the correct responses.
\end{itemize}

\subsection{API Schema}

\begin{itemize}
\item \textbf{Type Definitions:} Use type definitions to represent entities in the system, following a hierarchical structure.
\item \textbf{Queries and Mutations:} Use queries for fetching data and mutations for modifying data.
\item \textbf{Response Codes:} Although GraphQL doesn't use HTTP status codes in the same way as REST, ensure meaningful error messages are provided in the response.
\item \textbf{Request and Response Formats:} Use JSON as the default format for both request and response bodies.
\item \textbf{Filtering and Sorting:} Allow users to filter and sort results based on specific criteria within the query parameters.
\end{itemize}

\subsection{Authentication and Authorization}

\begin{itemize}
\item \textbf{Authentication:} Use token-based authentication (JWT) to authenticate users and secure API endpoints.
\item \textbf{Authorization:} Implement role-based access control to restrict access to certain queries and mutations based on user roles.
\item \textbf{Rate Limiting:} Implement rate limiting to prevent abuse and protect the API from excessive requests.
\end{itemize}
