\section{Definition of done} \label{definition_of_done}

\subsection{Overview}

The Definition of Done (DoD) is a comprehensive checklist that must be completed for each task, user story, or feature to be considered complete. This section defines the criteria that must be met for a task to be considered "done" for RadiantIQ development.

\subsection{Criteria}

\subsubsection{Code of conduct}
During or before code development every member of the team must take into account the following indications.
\begin{itemize}
	\item Every trance of code should be appropriately commented
	\item Every logical trance of the code should be documented in order to describe its logic
	\item Each element of the code should have a meaningful name and follow a coherent notation
	\item Every commit of a completed task should follow the adequate naming and git policy
	\item Each modification to existing code should originate from a task, or be a decision taken considering the whole team
	\item Every addition to the application should be discussed, approved, put in the project backlog and assigned to a sprint before being implemented
\end{itemize}

\subsubsection{Testing}
\begin{itemize}
    \item Unit tests have been written and cover all new and changed functionality.
    \item All unit tests passed.
    \item Integration tests have been performed to ensure new changes do not break existing functionality.
    \item Functional tests have been conducted to verify the feature works as expected.
    \item Regression tests have been run and passed to ensure existing functionality is not affected.
    \item Performance tests have been conducted, and performance metrics meet acceptable thresholds.
    \item No critical bugs remain unresolved.
	\item All tests in Github Actions are passing.
\end{itemize}

\subsubsection{Documentation}
\begin{itemize}
    \item Code is documented, including inline comments where necessary.
    \item Public APIs have clear and complete documentation.
    \item User-facing documentation has been updated (if applicable).
    \item Release notes have been updated to reflect the new changes.
\end{itemize}

\subsubsection{Security}
\begin{itemize}
    \item Security vulnerabilities have been pointed out.
    \item Dependencies have been reviewed and updated to ensure there are no security vulnerabilities.
    \item Data privacy concerns have been addressed.
\end{itemize}

\subsubsection{Deployment and Release}
\begin{itemize}
    \item The feature or fix has been deployed to a staging environment.
    \item All deployment scripts have been tested and are functioning correctly.
    \item The feature or fix has been demonstrated to stakeholders and feedback has been incorporated.
    \item Rollback plans have been prepared in case of deployment failure.
	\item Code is compatible with at least the top 3 most popular browsers
	\item Code is compatible with at least one mobile device (android based)
\end{itemize}


\subsection{Process}

\subsubsection{Review and Approval}
\begin{itemize}
    \item Each task or feature must undergo a code review process.
	\item Code respects the associated functional requirements completely
	\item Code runs in accordance with the specified non-functional requirements
    \item Reviewers must verify that all DoD criteria are met before approving the task.
\end{itemize}

\subsubsection{Quality Assurance}
\begin{itemize}
    \item Quality Assurance (QA) engineers will conduct testing to ensure all criteria are met.
    \item Any issues identified during testing must be resolved before the task is considered done.
\end{itemize}

\subsubsection{Deployment}
\begin{itemize}
    \item Ensure that the deployment pipeline is green and the feature is deployable.
    \item Verify that all necessary documentation and release notes are prepared.
    \item Verify that the feature has been deployed to Render and is functioning as expected.
\end{itemize}