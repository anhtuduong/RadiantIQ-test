\section{Definition of done} \label{definition_of_done}
The definition of done has been divided in two parts. The first is the code of conduct, which defines how the members of the team should develop the application. The second are the criteria for shippable, which are the conditions the developed version should respect in order to be considered complete and ready to be shipped to the customer.

\subsection{Code of conduct}
During or before code development every member of the team must take into account the following indications.
\begin{itemize}
	\item Every trance of code should be appropriately commented
	\item Every logical trance of the code should be documented in order to describe its logic
	\item Each element of the code should have a meaningful name and follow a coherent notation
	\item Every commit of a completed task should follow the adequate naming and git policy
	\item Each modification to existing code should originate from a task, or be a decision taken considering the whole team
	\item Every addition to the application should be discussed, approved, put in the project backlog and assigned to a sprint before being implemented
\end{itemize}

\subsection{Criteria for shippable}
Before delivery, each piece of code must pass through every one of the following indications
\begin{itemize}
	\item Code respects the associated functional requirements completely
	\item Code has been thoroughly tested and passes at least 90\% of unit tests
	\item Code runs in accordance with the specified non-functional requirements
	\item Code has been revised and reviewed by at least another member of the team
	\item Scan for security vulnerabilities must be completed
	\item Known insolvable weaknesses or bugs (even remote ones) should be recorded in a document and made public to the users and customers
	\item Code is compatible with at least the top 3 most popular browsers
	\item Code is compatible with at least one mobile device (android based)
	\item Code has been approved by customer
\end{itemize}